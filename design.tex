\chapter{Design\label{cha:design}}
Explain your design process.
Hint:  you will be employing elements from the textbook for Product Design and Development~\cite{ulrich2020product-design-development} and Axiomatic Design~\cite{suh2021des}.
Give a one-page explanation of how the processes works assuming the reader isn't familiar with them but has an engineering education.


\section{Requirements}
You now have a list of customer needs and benchmark metrics from competitors.
State assumptions, boundary Conditions, and/or constraints

\textbf{The rest of this section uses elements from Kristján Gerhard's MSc Thesis~\cite{gerhard2016suretrack}.
Please remember to replace the citations and information for your paper.}
It is time to map the relevant customer needs into Functional Requirements based upon what you learned from the benchmark chart.

As usual, start with a top-level to encapsulate your general strategy of needed functionality and the implementation:
The critical functionality for solving the problem is  \FR0: ``Contain \SI{25}{\kilogram} of fish on SureTrack conveyor until release is triggered''
Our tactic to implement this functionality is \DP0: Gable-reinforced stainless-steel locking bin with bi-directional discharge
\cite{gerhard2016suretrack}.


\section{Implementation}
Now decompose these functions further to generate Table~\ref{tab:first_level-frdp}.:
\begin{table}
  \center
  \caption{First level FR-DP mapping.~\cite{gerhard2016suretrack}}\label{tab:first_level-frdp}
  \begin{tabular}{lll} \toprule
    ID& Functional Requirement & Design Parameter \\ \midrule 
    1&Contain product&Main weldment\\
    2&Move product&Support system\\
    3&Discharge product &Discharge system\\
    \bottomrule
  \end{tabular}
\end{table}
Design Decomposition diagram goes here and explanation of what the arrows tell us.

From this diagram, chunk pieces together to make modules.
Explain what those modules are and what they do.

Put Incidental Interaction graph and related discussion

\section{\FR1 do something interesting}
For each requirement, explain with calculations and data how the related DP will meet the requirement..
This means each FR needs to have an associated test or tests (which will be checked in the Analysis chapter) and each DP needs to have a target metric which should be quantitative.
Put equations, data, sketches, drawings, CAD, and electrical schematics.

\section{\FR2 do something else}
For each requirement, explain with calculations and data how the related DP will meet the requirement..
This means each FR needs to have an associated test or tests (which will be checked in the Analysis chapter) and each DP needs to have a target metric which should be quantitative.
Put equations, data, sketches, drawings, CAD, and electrical schematics.


\section{\FR3 do something different}F
For each requirement, explain with calculations and data how the related DP will meet the requirement..
This means each FR needs to have an associated test or tests (which will be checked in the Analysis chapter) and each DP needs to have a target metric which should be quantitative.
Put equations, data, sketches, drawings, CAD, and electrical schematics.

\section{Summary}
Summarize the chapter and place a schematic of how all the modules fit together.

%%% Local Variables:
%%% mode: latex
%%% TeX-master: "main"
%%% End:
